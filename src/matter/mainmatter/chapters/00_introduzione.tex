\chptr{Introduzione alla Fisica}
\marginpar{\minitoc}

\section{Definizione e scopi della fisica}

Si possono formulare definizioni diverse riguardo la disciplina scientifica
della fisica, come la seguente:

\begin{tcolorbox}[colback = yellow!30, colframe = yellow!30!black, title = {Fisica}]
La fisica è lo studio quantitativo delle leggi fondamentali della natura, cioè
delle leggi che governano tutti i fenomeni naturali dell'universo.
\end{tcolorbox}

\noindent La fisica non si occupa in realtà di tutti gli aspetti dell'universo.
Essa di fatto si interessa solamente di ciò che è quantificabile, ovvero
esprimibile da numeri e oggetti matematici più complessi. Per tale motivo,
la matematica è lo strumento essenziale di qeusta scienza.

Giungendo al sodo, la fisica intende descrivere, e quando possibile giustificare,
con coerenza e formalità ciò che osserviamo quotidianamente: perché l'acqua
bolle, quale altezza raggiungerà una palla lanciata in alto con una certa
velocità, quali principi rendono un circuito elettrico efficiente dal punto
di vista energetico, come lanciare un satellite intorno a Giove e così via. La
fisica cerca di individuare innanzitutto delle leggi.

\begin{tcolorbox}[colback = yellow!30, colframe = yellow!30!black, title = {Legge}]
    Una legge fisica è una regolarità della natura esprimibile in forma
    matematica, ma anche una verità non dimostrabile (in tal caso si usa spesso il
    sinonimo ``principio'') che tuttavia non contraddice i
    fenomeni osservabili dell'esperienza.
\end{tcolorbox}

\noindent La fisica si avvale del metodo sceintifico, secondo cui la natura deve
essere interrogata per vie sperimentali, facendosi guidare da ipotesi e
modelli teorici. Una particolarità di questo metodo è la capacità di isolare
un certo fenomeno che si intende studiare, tralasciando (si userà spesso il
termine ``trascurare'') certi aspetti ritenuti non rilevanti in modo da
scoprire quelle regolarità dalle quali potrebbe essere dedotta una certa
relazione matematica.

Il ruolo della matematica è quello di fornire un linguaggio formale per descrivere
quantitativamente i fenomeni osservati e costruire modelli utili alla loro
trattazione.



\section{Grandezze fisiche}
La fisica è una scienza quantitativa, ovvero essa si occupa di caratteristiche
e proprietà del reale che possono essere misurate e quantificate: le cosiddette
grandezze fisiche. Esempi di grandezze fisiche sono la lunghezza, la massa, la
temperatura, la durata temporale e così via.

\begin{tcolorbox}[colback = yellow!30, colframe = yellow!30!black, title = {Grandezza fisica}]
Una grandezza fisica è una caratteristica di un oggetto o di un fenomeno che può
essere misurata in termini quantitativi (oltre che oggettivi, ovvero indipendentemente
dalle sensazioni personali degli individui).
\end{tcolorbox}

\noindent È implicito, intuitivamente, il concetto di misura. Misurare una grandezza
fisica significa confrontarla con una grandezza ``campione'', detta unità
di misura, e stabilire quante volte l'unità è contenuta nella
grandezza data. Il valore numerico ottenuto è la misura della grandezza e deve
essere sempre accompagnato dall'unità scelta.
In altre parole, la misura non è altro che un rapporto tra la
grandezza che si intende misurare e la grandezza campione scelta convenzionalmente
per tale scopo.

Mostriamo un esempio: supponiamo di voler misurare la lunghezza di qualsiasi cosa
in ``penne''\footnote{Tratto da una storia vera.}. Decidiamo poi di misurare l'altezza
di una porta utilizzando l'unità appena scelta. Supponiamo quindi di aver registrato il
seguente dato:
\[ h = 20 \text{ penne} \]
Notare come siano stati specificati:
\begin{itemize}
    \item Un nome per l'oggetto che si intendeva misurare, $h$, ovvero l'altezza
    della porta.
    \item Il valore numerico individuato, 20.
    \item Un'affermazione per legare il nome e il dato, = (``corrisponde a'', ``è
    uguale a'')—caratteristica che peraltro si trova anche nei linguaggi di
    programmazione. Si presti bene attenzione che l'uguaglianza matematica
    non implica necessariamente l'equivalenza fisica\footnote{Noteremo tale
    sottigliezza enunciando il principio zero della termodinamica, ma se non
    lo conoscete non preoccupatevi ora, c'è ancora molta strada da fare.}.
    \item L'unità di misura, ``penne''.
\end{itemize}
Tuttavia, tale misurazione non è stata affatto sincera: non vi è la
garanzia che il valore registrato sia esatto e, probabilmente, nel
misurare la porta avremmo affermato di aver misurato ``all'incirca'
un'altezza pari a 20 penne. La prossima sezione
tratta questo problema, ovvero quello dell'incertezza.

\subsubsection{Alcune grandezze fondamentali}
Tra le grandezze fondamentali studiate in questo corso ci sono:
\begin{itemize}
    \item La massa: intuitivamente, rappresenta la quantità di
    materia di cui un corpo, cioè un oggetto, è carico. Non si
    deve confondere con il peso, anche se la dinamica mostra che
    peso e massa sono correlate. L'effetto della massa è il seguente:
    più un oggetto è massivo, più
    esso è difficile da mettere in moto o da frenare, se in
    movimento (assumendo che esso non sia influenzato da altri
    ``agenti esterni'').

    \item La lughezza: intesa in senso unidimensionale, permette
    di quantificare la distanza tra due punti. Si può modellare
    un sistema fisico che vive in uno spazio scomponendolo in
    dimensioni più semplici, ognuna misurabile mediante lunghezze.
    Ad esempio, una stanza ha tre dimensioni, scomponibili in
    altezza, larghezza e lunghezza, misurabili per esempio in
    metri.

    \item Il tempo: supponiamo che il lettore abbia già un'idea
    intuitiva di questo concetto.
\end{itemize}

\noindent Gli standard internazionali definiscono altre grandezze
fondamentali, ma con queste sole tre se ne possono derivare moltissime
altre, dette composte, che si incontrano in fisica: velocità, forza,
pressione e così via.

\section{Incertezza}
Idealmente, si vorrebbe impiegare, grazie alle misure, numeri puntuali ed esatti.
In altre parole, dei numeri con una precisione indefinita, aventi un numero
illimitato di cifre decimali e non.

Ma quando si effettua una misura di una grandezza, il risultato ottenuto è noto solo
con una certa precisione. Riprendendo l'esempio della penna, è impossible
misurare con certezza tutte le lunghezze, in quanto non multipli esatti della
penna stessa: ci sarà sempre un certo margine di ``un pezzo di penna'',
minore dell'unità prescelta. Ma al di sotto di quella unità non è possibile
fornire alcuna garanzia sulla puntualità del dato. In altre parole, la
sensibilità\footnote{La più piccola variazione della grandezza che lo
strumento è in grado di rilevare.} dello strumento è uno dei limiti alla precisione
della misura. Allora, come nelle architetture di computer è presente un
numero limitato di bit per registro, anche le misure fisiche non possono
essere scritte con un numero infinito di cifre. Parliamo dunque di cifre
significative.

\begin{tcolorbox}[colback = yellow!30, colframe = yellow!30!black, title = {Cifre significative del risultato di una misura}]
Le cifre significative del risultato di una misura sono le cifre note
con certezza e la prima cifra incerta. In altre parole, esse sono le cifre che si
possono controllare con lo strumento impiegato nella misura.
\end{tcolorbox}

\noindent Ad esempio, il valore corrispondente alla lunghezza di una barca $L = 10.5$ m
possiede tre cifre significative, che non equivale a $10.50$ m. Il secondo dato,
infatti, dichiara che la misurazione è stata possibile controllando le cifre
fino al centimetro. $L = 0.002$ possiede solo una cifra significativa, perché
in genere si ignorano gli zeri a sinistra della prima cifra significativa diversa
da zero. Possono essere ambigui valori come $L = 2500 \text{ m}$: Quali zeri sono
cifre significative? Chi ha compiuto la misura può aver utilizzato un'asta lunga
un centinaio di metri, dunque non è possibile stabilire il valore delle cifre
meno significative delle centinaia. Come vedremo tra poco, è utile esprimere questi valori in
notazione scientifica per eliminare ambiguità.

Vi potrebbero anche essere errori dovuti a imprecisioni introdotte nell'utilizzo
degli strumenti di misura. Questo errore deve tuttavia essere quantificato ed ogni
misura ne è affetta (comprese quelle che non la riportano).

\begin{tcolorbox}[colback = yellow!30, colframe = yellow!30!black, title = {Risultato della misura di una grandezza}]
Il risultato della misura di una grandezza è sempre un'approssimazione
accompagnata da una certa incertezza, ovvero un \textbf{valore attendibile} $\overline{x}$
e un \textbf{errore assoluto} $e_x$ (o semplicemente \textbf{incertezza}).
\[ x = \overline{x} \pm e_x  \]
\end{tcolorbox}

Questo risultato non è quindi altro che un intervallo in cui il valore reale
della misura si trova. Ci limiteremo agli errori relativi a singole misure,
nelle quali $\overline{x}$ corrisponde al valore misurato e $e_x$ la sensibilità dello
strumento\footnote{Durante i calcoli, gli errori si propagano,
secondo definizioni che tuttavia non verranno affrontate.}. Di conseguenza, possiamo ora correggere il risultato della misura
effettuata in penne:

\[ H = (20 \pm 1) \text{ penne} \]

Negli esercizi non vedremo mai risultati scritti con la loro incertezza. L'unica
cosa alla quale è bene fare attenzione è gestire i calcoli di numeri con diverse cifre
significative: se i dati sono espressi con un numero differente di cifre significative
\begin{itemize}
    \item Si sconsiglia di effettuare calcoli parziali, ovvero effettuati a ``passi''.
    È buona pratica esprimere simbolicamente il risultato, per poi calcolarlo tutto d'un
    colpo sostituendo i valori.

    \item Generalmente, nei calcoli finali, il risultato ``eredita'' il numero di cifre
    significative del dato che ne possiede di meno.
\end{itemize}

\section{Notazione scientifica}
Unità di misura come il metro e il kilogrammo sono comode nella vita di tutti i
giorni, ma rappresentano quantità enormi su scala atomica e subatomica e quantità
minuscole su scala astronomica e cosmica. Conseguenza di ciò è che alcune misure
possono essere espresse da numeri ``scomodi''. Considerando solo valori attendibili,
la massa dell'atomo di idrogeno è circa
\[ m_H = 0.000 000 000 000 000 000 000 000 001 67 \text{ kg} \]
mentre la massa della Terra è
\[ m_T = 5,970,000,000,000,000,000,000,000 \text{ kg} \]
È pressoché evidente il motivo di tale scomodità: la notazione è di difficile
trattazione. Viene dunque in aiuto la \textbf{notazione scientifica}, ovvero una
notazione numerica che permette di contrarre rappresentazioni estese impiegando
potenze di 10. Nella notazione scientifica, ogni numero è scritto come prodotto
di due fattori:
\begin{itemize}
    \item Un numero decimale $x:x\in \mathbb{R}, 1\leq x < 10$[\footnote{In realtà, questa notazione corrisponde alla variante ``ingegneristica''. Esiste anche una notazione che prevede che il valore espresso $x$ sia $0\leq x < 1$.}].
    \item Una potenza di 10, con esponente intero.
\end{itemize}
Pertanto, le misure precedenti si possono esprimere in notazione scientifica come
segue:
\[ m_H = 1.67 \cdot 10^{-27} \text{ kg} \]
\[ m_T = 5.97 \cdot 10^{24} \text{ kg}\]
Notare come la notazione sia in grado di eliminare ambiguità sul numero di cifre
significative: ora sappiamo che la massa della Terra è stata calcolata fino a
tre cifre significative e non 25.

Non sempre è necessario calcolare esattamente il valore di una certa grandezza.
Talvolta basta averne solo un'idea approssimata. Supponiamo, ad esempio, che sia
sufficiente sapere se una certa massa vale all'incirca 1 grammo oppure 1
ettogrammo. In questo caso, possiamo accontentarci di stimare il valore della
massa con un'accuratezza di un fattore 10, cioè di calcolare il suo ordine di
grandezza.

\begin{tcolorbox}[colback = yellow!30, colframe = yellow!30!black, title = {Ordine di grandezza}]
L'ordine di grandezza di un numero è la potenza di 10 più vicina a quel numero.
\end{tcolorbox}

\noindent Per determinare l'ordine di grandezza di un numero occorre quindi esprimerlo in
notazione scientifica—prodotto di un numero decimale compreso tra 1 e 10 e di
una potenza di 10—e poi approssimare il valore alla potenza di 10 più vicina.
In particolare:
\begin{itemize}
    \item Se il numero decimale è minore di 5, si mantiene l'esponente della
    potenza. Ad esempio:
    \[ 3.6 \cdot 10^2 \to \text{ Ordine di grandezza } 10^2 \]
    \[ 4.2 \cdot 10^{-3} \to \text{ Ordine di grandezza } 10^{-3} \]

    \item Se il numero decimale è maggiore di 5, si somma +1 all'esponente della
    potenza. Ad esempio:
    \[ 9 \cdot 10^2 \approx 10 \cdot 10^2 \to \text{ Ordine di grandezza } 10^3 \]
    \[ 8.1 \cdot 10^{-12} \approx 10 \cdot 10^{-12} \to \text{ Ordine di grandezza } 10^{-11} \]
\end{itemize}

\marginpar{\begin{center}
    \begin{tabular}{c|c|c}
        Potenza   & Simbolo & Prefisso\\
        \hline
        $10^{12}$  & T       & Tera\\
        $10^{9}$   & G       & Giga\\
        $10^{6}$   & M       & Mega\\
        $10^{3}$   & k       & kilo\\
        $10^{-3}$  & m       & milli\\
        $10^{-6}$  & $\mu$   & micro\\
        $10^{-9}$  & n       & nano\\
        $10^{-12}$ & p       & pico\\
    \end{tabular}
\end{center}}

\noindent Sono stati definiti dei prefissi stadard per certi ordini di grandezza notevoli,
cioè quelli che, escludendo la potenza nulla, rappresentano multipli di tre.
Utilizzando questi prefissi, di fianco all'unità di misura adottata, si contrae
ancora di più la notazione scientifica, sottointendendo un certo ordine di
grandezza.


%\section{Approfondimenti}
%La fisica trae le proprie origini dalla ``filosofia naturale'',
%prima della nascita della scienza moderna. La fisica di oggi ha
%ancora alcuni segni di

%questione legge = qualcosa di regolare e che finora non abbiamo
%mai visto infrangersi