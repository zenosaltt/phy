\chapter*{Guida al testo}
Saremo onesti e concisi: questi appunti sono lunghi e prolissi.
Ma alcuni aspetti sono stati curati per soddisfare le esigenze di
coloro che intendono solo dare rapide occhiate alle nozioni di rilievo
in questo corso.

\begin{itemize}
    \item \textit{Indice}: l'indice è la prima arma a portata di studenti che intendono
    consultare rapidamente gli appunti. Ci siamo impegnati al meglio per rendere
    concisi i titoli di capitoli e sezioni per raggiungere questo scopo. L'indice
    contiene collegamenti cliccabili che conducono direttamente alla pagina selezionata.

    \item \textit{Microindici}: all'inizio di ogni capitolo viene collocato un microindice
    contenente la lista di macrosezioni, visibile nel margine destro. Anche questi microindici
    sono dotati di collegamenti interni alle rispettive pagine del testo.

    \item \textit{Box colorati}: definizioni, leggi e principi notevoli sono risaltati da box colorati.
    
    \item \textit{Equazioni etichettate}: se non evidenziate dai box, le leggi e altri
    risultati (per lo più matematici) rilevanti sono comunque numerati tra
    parentesi tonde.

    \item \textit{Appendici (in sviluppo...)}: vengono anche curate alcune appendici
    che riportano risorse utili di svariato genere, quali un piccolo compendio di esercizi,
    le leggi fisiche rilevanti in questo corso e altro ancora. L'indice riporta anche
    queste appendici.
\end{itemize}

Per coloro che invece manifestano maggiore interesse per questa scienza, alcuni
capitoli includono approfondimenti che possono stuzzicare menti impavide (anche
se non sono di certo questi appunti a contenere tutti quelli più interessanti,
qui troverete solo un minimo assaggio). Gli approfondimenti non sono
indispensabili per questo corso.

\section*{Requisiti}
Seppur superficialmente, questi appunti coprono un'ampia area della fisica
e il substrato matematico necessario alla sua comprensione è piuttosto eterogeneo.
Anche se non sarà sempre necessario ricordare tutto,
questi appunti presuppongono conoscenze di: analisi matematica (derivazione,
integrazione), trigonometria, geometria analitica (sistemi di assi
cartesiani e calcoli annessi) geometria e algebra lineare (vettori,
sistemi di equazioni). Conoscenze matematiche più avanzate
sono di aiuto ma non strettamente necessarie.

\section*{Sulla notazione}
Si utilizzano spesso in questo testo notazioni matematiche compatte.
Invitiamo il lettore a consultare l'appendice dei simboli per eventuali
chiarimenti. I testi potrebbero non essere esenti da abusi di notazione\footnote{Scusateci tanto
ma il font AMS è talmente bello che è facile perdere il controllo scrivendo
appunti in \LaTeX.}.

%Un intero capitolo di approfondimento contiene argomenti
%relativi all'Elettromagnetismo, non più mostrato, come un tempo, durante
%l'anno accademico attuale (2023/2024).