\section*{Riconoscimenti}
Questi appunti non sono stati realizzati dal nulla.
Alcuni materiali di supporto sono stati impiegati
per portare alla luce queste pagine.

\subsubsection*{Testi}
Gran parte del testo è una trascrizione in linguaggio tipografico \LaTeX\@
degli appunti reperiti durante
il corso di Fisica (a.a. 2023-2024) tenuto dal prof. Roberto Iuppa, presso
l'Università degli Studi di Trento. Organizzazione e ordine dei
capitoli ricalcano la successione delle lezioni frontali, con alcune
rivisitazioni dell'ordine di esposizione degli argomenti.

Di grande ispirazione e supporto sono stati:
\begin{itemize}
    \item R. P. Feynman, R. B. Leighton, M. Sands. \textit{The Feynman Lectures on Physics. Volume I: Mainly Mechanics, Radiation, and Heat}. Addison-Wesley. 1964.
    \item J. Walker. \textit{Dalla Meccanica alla Fisica Moderna. Meccanica - Termodinamica}. Pearson. 2012.
\end{itemize}

\subsubsection*{Immagini}
Tutte le immagini che si trovano in questi appunti sono state realizzate
integralmente a mano, incluso il design di copertina, utilizzando
lo strumento Disegni della suite Google. Tutte le risorse
grafiche vettoriali di questo testo sono incluse nella repository del progetto.

\subsubsection*{Esercizi}
Gli esercizi proposti sono una selezione di problemi
giudicati interessanti o
di importanza basilare, perché incorporano principi chiave studiati nel corso.
Questi esercizi sono stati reperiti dai libri di testo citati precedentemente
e da fogli relativi ad esercitazioni dell'anno accademico corrente (a loro
volta raccolti dal Web), più alcuni testi d'esame antecedenti l'anno accademico
nel quale questa dispensa è stata redatta per la prima volta. Questi testi
d'esame sono stati pubblicati dal professore stesso sul canale Moodle del corso.