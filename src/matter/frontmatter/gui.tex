\chapter*{Guida al testo}
Saremo onesti e concisi: Questi appunti sono lunghi e prolissi.
Ma alcuni aspetti sono
stati curati per soddisfare le esigenze di coloro che intendono solo dare rapide
occhiate alle nozioni di questo corso.

\begin{itemize}
    \item \textit{Indice}: l'indice è la prima arma a portata di studenti che intendono
    consultare rapidamente gli appunti. Ci siamo impegnati al meglio per rendere
    concisi i titoli di capitoli e sezioni per raggiungere questo scopo. L'indice
    contiene collegamenti cliccabili che conducono direttamente alla pagina selezionata.

    \item \textit{Microindici}: all'inizio di ogni capitolo viene collocato un microindice
    contenente la lista di macrosezioni, visibile nel margine destro. Anche questi microindici sono dotati di
    collegamenti (cliccabili) interni alle rispettive pagine del testo.

    \item \textit{Box colorati}: definizioni, leggi e principi notevoli sono risaltati
    da box colorati.
    
    \item \textit{Equazioni etichettate}: se non evidenziate dai box, le leggi e altri
    risultati (per lo più matematici) rilevanti sono comunque numerati tra
    parentesi tonde.

    \item \textit{Appendici}: vengono anche curate alcune appendici che riportano risorse
    utili di svariato genere, quali un piccolo compendio di esercizi, le leggi
    fisiche rilevanti in questo corso e altro ancora. L'indice riporta anche
    queste appendici.
\end{itemize}

Per coloro che invece manifestano maggiore interesse per questa scienza, alcuni
capitoli includono approfondimenti che possono stuzzicare conoscenze impavide (anche
se non sono di certo questi appunti a contenere tutti quelli più interessanti,
qui troverete solo un minimo assaggio).
Gli approfondimenti sono solo stati citati o addirittura ignorati durante
le lezioni e non sono conoscenze indispensabili.

\section*{Requisiti}
Questi appunti coprono un'area ristrettisima della fisica, ma il substrato
matematico necessario alla sua comprensione è piuttosto vasto. Anche
se non sarà sempre necessario ricordare questi strumenti nel dettaglio,
questi appunti presuppongono conoscenze di: analisi matematica (derivazione,
integrazione), trigonometria, geometria analitica (sistemi di assi
cartesiani e calcoli annessi) geometria e algebra lineare (vettori,
sistemi di equazioni). Conoscenze matematiche più avanzate di analisi
sono di aiuto ma non strettamente necessarie.

\section*{Sulla notazione}
Si utilizzano spesso in questo testo notazioni matematiche compatte.
Invitiamo il lettore a consultare l'appendice dei simboli per eventuali
chiarimenti. Sottolineamo che tra i potenziali errori di questa dispensa
possono essere presenti anche abusi di notazione.

%Un intero capitolo di approfondimento contiene argomenti
%relativi all'Elettromagnetismo, non più mostrato, come un tempo, durante
%l'anno accademico attuale (2023/2024).