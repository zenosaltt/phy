\section*{Errata}
È statisticamente difficile, se non impossibile, produrre un testo
privo di errori. Questi appunti non sono un'eccezione.
Spieghiamo di seguito come e cosa si può segnalare.

\subsubsection*{Come segnalare}
\begin{itemize}
    \item Se conosci gli autori, contattandoli o cercandoli direttamente in Università
    o su canali di comunicazione come Telegram o E-Mail istituzionale
    (\texttt{<nome>.<cognome>@studenti.unitn.it}).

    \item Se conosci GitHub, aprendo una issue sulla repo (\href{https://github.com/zenosalty/courses-phy}{\faGithub \space Link}).

    %\item Se conosci GitHub e in più \LaTeX\@, mettendo mano direttamente al codice
    %e proponendo delle modifiche mediante una pull request.
\end{itemize}

\subsubsection*{Cosa segnalare}
Dividendo gli errori in categorie, secondo un ordine di priorità
decrescente:

\begin{itemize}
    \item Errori grammaticali, lessicali, sintattici e tutti quegli errori nell'impiego
    del linguaggio che ostacolano la comprensione del testo.

    \item Nozioni che non corrispondono al vero o incomplete, di qualsiasi genere (ma
    attinenti alla materia trattata in queste pagine):
    leggi mal formulate; affermazioni, supposizioni, definizioni imprecise, false o
    superficiali\footnote{Sono ovviamente contemplate correzioni e integrazioni da parte di appasionati
    o esperti in campi specifici: la meccanica dei motori termici descritta brevemente
    nel capitolo di termodinamica; storia e filosofia; balistica e molto altro ancora.}; affermazioni false
    relative a fatti o persone reali.

    \item Errori di calcolo o risultati errati nelle equazioni e negli esempi del testo e negli
    esercizi dell'appendice.

    \item Irregolarità nell'utilizzo di notazioni standard, come i simboli matematici
    o la citazione di testi.

    \item Link malfunzionanti.
    
    \item Altri errori di battitura oppure grafici ed estetici.
\end{itemize}

Vengono contemplati con riguardo anche eventuali miglioramenti o integrazioni,
qualora il tempo e le energie a disposizione lo permettano:

\begin{itemize}
    \item Esercizi.
    \item Immagini che arricchiscono il testo e a supporto della comprensione.
    \item Rivisitazioni dell'ordine dei capitoli, delle sezioni, dei paragrafi, della veste grafica.
    \item Approfondimenti inerenti agli argomenti affrontati nel corso.
\end{itemize}