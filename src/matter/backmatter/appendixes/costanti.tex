\chptr{Costanti}

\vspace*{0.5cm}
La fonte dei valori delle costanti è la calcolatrice CASIO
\textit{fx-991ES PLUS}.

\begin{align*}
    &g      & \text{Accelerazione di gravità (superficie terrestre)} && 9.80665 \text{ m}/\text{s}^2\\
    &G      & \text{Costante di gravitazione universale}             && 6.67428 \times 10^{-11} \text{ Nm}^2/\text{kg}^2\\
    &\sigma & \text{Costante di Stefan-Boltzmann}                    && 5.6704 \times 10^{-8} \text{W}/(\text{m}^2\text{K}^4)\\
    &k_B    & \text{Costante di Boltzmann}                           && 000\\
    &R      & \text{Costante dei gas ideali}                         && 000\\
    &N_A    & \text{Numero di Avogadro}                              && 6.022 \times 10^{23}\\
    &c      & \text{Velocità della luce nel vuoto}                   && 299,792,458 \text{ m}/\text{s}
\end{align*}

\section*{Note}
\begin{description}
    \item[Curiosità: Lettere maiuscole] Vi siete mai chiesti perché
    i simboli di alcune unità di misura sono in maiuscolo, come
    W (Watt), N (Newton) e Pa (Pascal)? Nella
    stragrande maggioranza dei casi, ciò si deve al fatto che l'unità
    è dedicata ad una persona realmente esistita e dunque è doveroso utilizzare una sua iniziale con la
    prima lettera maiuscola.

    \item[Sulla costante di Stefan-Boltzmann] Ricordare questa costante
    fino alle primissime cifre significative è una passeggiata: si
    tratta di ricordare la sequenza ``5678'', con l'8 all'esponente
    e negativo.

    \item[Curiosità: Velocità della luce] Fizeau misurò la luce in
    maniera assai bizzarra, mediante un complesso meccanismo costituito
    da una ruota dentata e specchi, ricorrendo a precisione
    e tempistiche finissime.

    \item[Curiosità: Numero di Avogadro] Nonostante il nome, Avogadro
    non calcolò questo numero. Jean Baptiste Perrin effettuò misure
    sperimentali che condussero successivamente al valore oggi accettato,
    ma pure Albert Einstein, mediante studi sul moto browniano, diede
    importanti contributi alla stima di questa costante. Molte altre
    leggi e costanti che portano nomi di persone hanno avuto storie
    inaspettate, simili a questa.
\end{description}