\chptr{Lista dei simboli}

\section{Sulla notazione vettoriale}
Il testo rappresenta i vettori con lettere maiuscole o minuscole,
italiche e in grassetto. Tale scelta è stata adottata in quanto
standardizzata e più pratica da impiegare su carta. Questa notazione
è equivalente a quella utilizzata generalmente alla lavagna, dove
le lettere sono sovrastate da una freccia che punta verso destra.

\begin{center}
    $\vecsymb{v}$ equivale a $\vec{v}$
\end{center}

\noindent Se la lettera non è grassettata e il contesto di interpretazione
non è ambiguo, si intende il \textit{modulo} del vettore.

\begin{center}
    $v$ equivale a $|\vec{v}|$
\end{center}


\section{Simboli}

\begin{align*}
    &\therefore  & \text{Quindi}\\
    &F           & \text{Forza, modulo}\\
    &\vecsymb{F} & \text{Forza, vettore}\\
    &\omega      & \text{Velocità angolare}\\
    &p           & \text{Quantità di moto, modulo}\\
    &\vecsymb{p} & \text{Quantità di moto, vettore}\\
    &\vecsymb{I} & \text{Impulso, vettore}
\end{align*}