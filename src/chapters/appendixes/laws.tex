\chptr{Leggi fisiche notevoli}

\section{Legge di Hooke}
La legge di Hooke è una legge empirica che pone in relazione la forza
di richiamo esercitata da una molla e l'estensione della deformazione
che genera quella stessa forza.

\begin{align}
    \vecsymb{F} = -k \Delta\vecsymb{x}\label{hooke}
\end{align}

\noindent Si suppone che la deformazione $\Delta\vecsymb{x}$ avvenga sulla
stessa retta sulla quale la molla giace.


\section{Legge di Stefan-Boltzmann}

\begin{align}
    \varepsilon = \sigma e T^4\label{stefanboltzmann}
\end{align}

\section{Legge di Newton gravitazione universale}


\begin{align}
    \vecsymb{F} = -G \frac{m M}{\vecsymb{r}^2}\hat{r}
\end{align}

\section{Legge di Avogadro}

\begin{align}
    N = \frac{1}{k_B} \frac{pV}{T}
\end{align}

\section{Legge di Boyle}
\begin{align}
    p \propto \frac{1}{V}
\end{align}

\section{Equazione armonica}

\begin{align}
    \frac{d^2\vecsymb{x}}{dt^2} + \omega^2\vecsymb{x} = 0\label{harmony}
\end{align}

\section{Oscillatore a molla}

\begin{align}
    T = 2\pi \sqrt{\frac{m}{k}}
\end{align}

\section{Pendolo}

\begin{align}
    T = 2\pi \sqrt{\frac{l}{g}}
\end{align}