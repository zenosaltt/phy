\chptr{Relatività del moto}
\marginpar{\minitoc}

\epigraph{\emph{That view of things would be normal for me if I normally walked on my hands.}}{\href{https://www.youtube.com/watch?v=bJMYoj4hHqU}{\textcolor{blue}{\textit{Frames of Reference (1960)}}}}

Cominciamo questo capitolo ponendo al lettore una domanda:

\begin{center}
    \textit{Sei aristotelica/o oppure copernicana/o?}
\end{center}

\noindent Senza soffermarci troppo su cosa essa intenda,
sottolineamo brevemente che ciò di cui si tratta in questo capitolo
è stato uno dei temi più controversi nella storia della fisica, quello
che forse ha sconvolto di più convinzioni un tempo ben radicate e
che ha infiammato dibattiti che hanno pure fatto la storia della
letteratura\footnote{\href{https://youtu.be/0kxarmulkiA?feature=shared&t=6180}{\textcolor{blue}{\textit{Marco Paolini - ITIS Galileo}}} (circa 10 minuti a partire dal tempo 1:43:00).}. E non è tutto qui: Gli ultimi grandi sviluppi della relatività
del moto sono avvenuti poco più di un secolo fa ad opera di Lorentz, Einstein
e altri ancora e i cambiamenti da loro apportati hanno condotto a
stravolgimenti ancora più bizzarri. Perché ciò che questo ramo della
fisica rivela è che più osserviamo con attenzione la realtà, più essa non
è come un attimo prima poteva sembrare. Per tale motivo la relatività può
essere allo stesso tempo semplice e complessa, perché offre una visione
del tutto insolita di ciò che ci appare come ``normale''.


\section{Sistemi di riferimento}
``Relativo'' è un termine che ben si sposa con l'espressione ``a qualcosa''.
Quel qualcosa viene definito nelle teorie relativistiche della fisica
\textit{sistema di riferimento}, che intuitivamente rappresenta il punto di
vista dal quale si effettuano delle misurazioni o semplici osservazioni di
fenomeni.

\section{Principio di relatività galileiana}
\section{Forze apparenti}

\subsection{L'ascensore}

\section{Approfondimento: RR}
Questa sezione è interamente dedicata alla relatività ristretta, sviluppata
nelle sue forme più celebri da Einstein nei primi lustri del Ventesimo secolo.
Vogliamo dare un assaggio di questa teoria per vari motivi: Si tratta in primo
luogo di fisica classica (uno dei motivi è l'impiego degli stessi strumenti
matematici); essa è inoltre un buon esempio di ridefinizione e affinamento
radicali di teorie precedenti, innanzitutto la relatività galileiana, e convinzioni
comuni.

\subsection{I principi della relatività ristretta}