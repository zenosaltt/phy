\chptr{Meccanica e Urti}
\marginpar{\minitoc}

Vogliamo prevedere lo stato del sistema dopo l'urto, in termini di
velocità (vettori) dei corpi coinvolti. Ci serve un \textit{sistema
isolato} e la terza legge della dinamica.

\section{Quantità di moto}
Abbiamo sempre espresso la seconda legge come
\[ \mathbf{F} = m\mathbf{a} = m\frac{d\mathbf{v}}{dt} \]
ma questa proposizione afferma che l'effetto dell'agente esterno (la forza
$\mathbf{F}$) si traduce interamente in una variazione dello stato di
moto del corpo (accelerazione $\mathbf{a}$). Si suppone quindi che la
massa sia sempre costante, anche se ciò non è sempre vero. Ad esempio,
un razzo pieno di carburante non avrà la stessa massa che aveva in partenza
una volta arrivato in orbita, quindi la forza esercitata dalla propulsione
dei motori si è tradotta non solo in una variazione dello stato di moto,
ma anche in una variaizione della massa. Non tratteremo sistemi complessi
come il razzo, ma ciò fa intuire che la seconda legge della dinamica può
essere generalizzata nella forma seguente
\begin{align}
    \mathbf{F} = \frac{d}{dt}(m\mathbf{v}) = \frac{d\mathbf{p}}{dt}
\end{align}
Dove la quantità $\mathbf{p}$ prende il nome di \textit{quantià di moto},
definita come
\begin{align}
    \mathbf{p} \eqdef m\mathbf{v}
\end{align}
Come dice il termine, la quantità di moto descrive il moto dei corpi
sulla base della velocità di una massa più la massa stessa, a differenza
di quanto accade nello studio cinematico del moto, dove solo le variazioni
dello stato di moto contano, slegate da cause (forze) e materia (massa).

\section{Il fenomeno dell'urto}
\vspace{8pt}
\begin{tcolorbox}[colback = yellow!30, colframe = yellow!30!black, title = {Urto}]
    Un urto è un'interazione tra corpi, nella quale
\end{tcolorbox}
\vspace{5pt}

Un classico esempio che unisce nozioni su urti e quantità di moto è il
tavolo da biliardo. Supponiamo di avere due palle, 1 e 2, sul tavolo in moto
rettilineo
uniforme e in rotta di collisione tra loro; ovvero, sappiamo con certezza
che la loro traiettoria si intersecherà e che tale punto verrà raggiunto
da entrambi i corpi nel medesimo istante di tempo. L'esprerienza ci permette
di concludere che, passata la \textit{zona d'urto}, le palle non procederanno
sulle stesse rette dei moti precedenti, ma devieranno.
Dobbiamo fare alcune assunzioni fondamentali. Innanzitutto, supporremo che
nessun altro agente agirà sul sistema appena descritto (aiuterebbe immaginare
due palle che vagano nello spazio profondo).
Immaginiamo l'intervallo temporale nel quale le due palle saranno a contatto
tra loro, collidendo: entrambe eserciteranno una forza sull'altra e aiutati
dalla terza legge della dinamica sappiamo che
\[ \textbf{F}_{1 \to 2} = -\textbf{F}_{2 \to 1} \]
cioè l'applicazione di una forza su una palla determina una forza identica
in modulo e direzione, ma verso opposto e applicata sull'altra palla.
Sviluppiamo l'equazione sfruttando la definizione di quantità di moto,
ricordando che una forza $x \to y$ determina una variazione dello stato di
moto di $y$.
\[ \frac{d\mathbf{p}_2}{dt} = -\frac{d\mathbf{p}_1}{dt} \]
Ricorrendo agli usuali abusi di notazione matematica, ma ragionevoli da
un punto di vista fisico, semplifichiamo l'intervallo di tempo infinitesimale
del differenziale:
\begin{align*}
    &d\mathbf{p}_2 = -d\mathbf{p}_1\\
    &d[\mathbf{p}_1 + \mathbf{p}_2] = 0\\
    &d\mathbf{p}_\text{tot} = 0
\end{align*}
Abbiamo mostrato, in anticipo, che per un sistema di due corpi come le palle
da biliardo, assumendo che non agiscano forze esterne, la quantità di moto
totale del sistema si conserva. Come l'energia meccanica, possiamo dunque
concludere che un urto ideale non modifichi la quantità di moto del sistema.
Un'altra supposizione importante che è stata
sottointesa è la seguente: le distanze (eventuali variazioni della forma
degli oggetti) e i tempi (intervalli di tempo nei quali avviene il contatto
o più generalmente l'azione delle forze generate nella collisione) tipici dell'urto sono
molto piccoli e trascurabili rispetto a quelli normalmente osservabili al
di fuori dell'urto.
Con tutte queste assunzioni possiamo descrivere il
sistema attraverso la seguente approssimazione (dove $m$ indica la massa
di una palla), che consente di descrivere lo stato del sistema prima o
dopo l'urto, \textit{iniziale} e \textit{finale}.
\[ m_1v_{1,i} + m_2v_{2,i} = m_1v_{1,f} + m_2v_{2,f} \]

\section{Urto elastico}
In un urto elastico

\section{Forze interne ed esterne}
Approfondiamo il concetto di \textit{sistema di punti materiali} e studiamone
uno contentente un certo numero di punti materiali $N$. Immaginiamo che
tra questi punti agiscano forze di varia natura: repulsive elettriche,
attrattive gravitazionali ecc.; inoltre, supponiamo che vengano applicate
altre forze dall'esterno di questo sistema di punti materiali\footnote{Una
immagine esplicativa è il polmone, dove le molecole dell'aria formano i
punti materiali e i muscoli del torace sono gli agenti esterni.}.
Possiamo suddividere le forze in gioco in due insiemi:
\begin{enumerate}
    \item \textbf{Forze interne}: le forze che i punti esercitano gli uni
    sugli altri. Forze che descrivono l'interazione tra i punti.
    \item \textbf{Forze esterne}: le forze che l'ambiente esterno esercita
    sul sistema, l'insieme di punti.
\end{enumerate}
Ogni punto $i$-esimo sarà sottoposto ad una certa forza totale, o risultante, derivante
dalla somma/sovrapposizione di tutte le forze precedentemente descritte
\begin{align*}
    \mathbf{R}_i = m_i\mathbf{a}_i\\
    \mathbf{R}^{(E)}_i + \mathbf{R}^{(I)}_i = m_i\mathbf{a}_i
\end{align*}
Definiamo le risultanti delle forze esterne, che supponiamo essere presenti
in un certo numero $M$, ed interne agenti sul punto
$i$-esimo:
\begin{align*}
    \mathbf{R}^{(E)}_i &\eqdef \sum_{k = 1}^{M} \mathbf{F}^{(E)}_{k \to i}\\
    \mathbf{R}^{(I)}_i &\eqdef \sum_{j = 1}^{N} \mathbf{F}^{(I)}_{j \to i} \qquad j \not = i
\end{align*}
Supponiamo che un punto non eserciti alcuna forza su se stesso.
La risultante di tutte le forze in gioco sarà
\[ \mathbf{R} = \sum_{i = 1}^{N} \mathbf{R}_i \]
In tale somma, concentriamoci sulla risultante delle forze interne:
\[ \mathbf{R}^{(I)} = \sum_{i = 1}^{N} \mathbf{R}^{(I)}_i = \sum_{i = 1}^{N} \sum_{j = 1}^{N} \mathbf{F}^{(I)}_{j \to i} \]
In questa somma, supponiamo che non vi siano forze agenti su un corpo
ed esercitate dal corpo stesso, dunque poniamo $\mathbf{F}^{(I)}_{j \to i} = \overrightarrow{0} \quad \forall j = i$.
Sappiamo che vale la terza legge della dinamica, dunque
$\mathbf{F}^{(I)}_{j \to i} = -\mathbf{F}^{(I)}_{i \to j}$. Ma allora
\begin{align}
    \mathbf{R}^{(I)} = \sum_{i = 1}^{N} \sum_{j = 1}^{N} \mathbf{F}^{(I)}_{j \to i} = \overrightarrow{0}\label{interne}
\end{align}
Abbiamo appena dimostrato, grazie all'ipotesi della terza legge,
che, in un sistema di punti materiali, la risultante delle forze interne
è nulla.

\section{Punto materiale}
Dall'Equazione \ref{interne} possiamo dedurre che la risultante delle
forze, interne ed esterne, coinvolte in un sistema di punti materiali
è determinata solamente dalle forze esterne.
\[ \mathbf{R} = \mathbf{R}^{(E)} = \sum_{i} \mathbf{R}^{(E)}_i = \sum_i m_i\mathbf{a}_i \]
Dalla precedente equazione, si può ricavare un'interessante definizione:
\[ \sum_i m_i\mathbf{a}_i = \sum_i m_i \frac{d^2\mathbf{x}_i}{dt^2} = \left(\sum_i m_i\right) \frac{d^2}{dt^2}\left[ \frac{\sum_i m_i\mathbf{x}_i}{\sum_i m_i} \right] \]
Abbiamo ottenuto un termine molto interessante, un artificio matematico
che ha interpretazioni e applicazioni piuttosto importanti: \textit{il
centro di massa}.
\begin{align}
    \mathbf{x}_\text{CM} \eqdef \frac{\sum_i m_i\mathbf{x}_i}{\sum_i m_i}
\end{align}
Dalla definizione è banale ricavare la velocità e l'accelerazione del
centro di massa. Riprendendo le equazioni precedenti e ponendo $M = \sum_i m_i$,
possiamo concludere che
\[ \mathbf{R}^{(E)} = M\frac{d^2\mathbf{x}_\text{CM}}{dt^2} = M\mathbf{a}_\text{CM} \]

\section{Impulso}
\[ \left\langle \mathbf{F} \right\rangle = \mathbf{F}_\text{impulsiva} \simeq \frac{\Delta\mathbf{p}}{\Delta t} \]
\[ \Delta\mathbf{p} = \mathbf{F}_\text{impulsiva}\Delta t \]
Deformazioni per tempi brevi, tempo di contatto.
\[ \Delta\mathbf{p} = m\mathbf{v}_f - m(\overrightarrow{0}) \qquad F_\text{imp} = \frac{mv_f}{T} \]
