\chapter*{Guida al testo}
Saremo onesti e concisi: Questi appunti sono lunghi e prolissi (facciamo presente
alla lettrice o al lettore che, se ella o egli sta leggendo questa sezione,
non ci troviamo ancora all'indice). Ma tutti i fini che ci prefiggiamo sono ben
giustificati.

Cominciamo dal primo: La pazienza. La fisica non è una scienza come tutte le altre;
è un pensiero dalle radici filosofiche e devoto alla comprensione totale della
realtà, intesa nei suoi aspetti quantitativi e misurabili. Dunque, non è solo la
matematica che rende questa materia apparentemente così complessa, ma è prima di
tutto la sua visione del mondo. Se nel passato si è avuto poco a che fare con la
fisica, è necessaria pazienza per adottare questa visione. La fisica fonda sulla
pazienza, ovvero il metodo scientifico: Provare e riprovare, imboccare una strada
con in mano un'ipotesi per poi tornare sui propri passi se i presupposti non sono
corretti. E osservare, osservare con attenzione.

Il secondo fine: La curiosità. Se non si è cursiosi in una materia, scientifica o
non, il suo studio perde di valore, privo di interesse. Parimenti la fisica richiede
grande curiosità, perché tramite essa sorgono domande, che alimentano la curiosità,
che generano altre domande, che conducono a ipotesi, a modelli della realtà che,
mano a mano che si sviluppano, sono in grado di rispondere, almeno in parte, a
quelle domande. La fisica non è fatta di sole teorie immutabili; è frutto di rivoluzioni
e persone che per curiosità hanno messo in dubbio molte convinzioni su quello
che conosciamo. E per questo quelle persone hanno permesso a noi di conoscere di più.

Il terzo fine: L'immaginazione. Molte delle teorie appartenenti alla fisica
sono frutto di intuizioni e processi creativi. Ma il difetto della scienza è
di non ammettere questa ``debolezza'', perché ad ogni teoria deve essere
accompagnata una buona giustificazione, una solida base matematico-quantitativa.
Sappia chi sta leggendo che la fisica, anche se lo nasconde con vergogna, è
nulla senza immagniazione. Non fatevi ingannare dai rigidi formalismi che si
incontrano negli studi di questa scienza, perché Clausius, Carnot, Newton,
Galileo e molti altri citati o non citati in questa dispensa hanno immaginato,
intuito. La domanda ``E se...?''\footnote{xkcd} dovrebbe essere incisa nella mente
di chi studia fisica, perché è il timone della curiosità.

Esposti i fini di questi appunti, 